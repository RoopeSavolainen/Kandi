%%%%%%%%%%%%%%%%%%%%%%%%%%%%%%%%%%%%%%%%%%%%%%%%%%%%%%%%%%%%%%%%%%%%%%%%%%%%%%%%
%%%%%%%%%%%%%%%%%%%%%%%%%%%%%%%%%%%%%%%%%%%%%%%%%%%%%%%%%%%%%%%%%%%%%%%%%%%%%%%%
%%                                                                            %%
%% thesistemplate.tex version 3.01 (2017/10/06)                               %%
%% The LaTeX template file to be used with the aaltothesis.sty (version 3.01) %%
%% style file.                                                                %%
%%                                                                            %%
%% This is licensed under the terms of the MIT license below.                 %%
%%                                                                            %%
%% Copyright 2017, by Luis R.J. Costa, luis.costa@aalto.fi,                   %%
%% Copyright 2017 documentation in Finnish in the template by Perttu Puska,   %%
%% perttu.puska@aalto.fi                                                      %%
%% Copyright Swedish translations 2014 by Elisabeth Nyberg,                   %%
%% elisabeth.nyberg@aalto.fi and Henrik Wallén, henrik.wallen@aalto.fi        %%
%%                                                                            %%
%% Permission is hereby granted, free of charge, to any person obtaining a    %%
%% copy of this software and associated documentation files (the "Software"), %%
%% to deal in the Software without restriction, including without limitation  %%
%% the rights to use, copy, modify, merge, publish, distribute, sublicense,   %%
%% and/or sell copies of the Software, and to permit persons to whom the      %%
%% Software is furnished to do so, subject to the following conditions:       %%
%% The above copyright notice and this permission notice shall be included in %%
%% all copies or substantial portions of the Software.                        %%
%% THE SOFTWARE IS PROVIDED "AS IS", WITHOUT WARRANTY OF ANY KIND, EXPRESS OR %%
%% IMPLIED, INCLUDING BUT NOT LIMITED TO THE WARRANTIES OF MERCHANTABILITY,   %%
%% FITNESS FOR A PARTICULAR PURPOSE AND NONINFRINGEMENT. IN NO EVENT SHALL    %%
%% THE AUTHORS OR COPYRIGHT HOLDERS BE LIABLE FOR ANY CLAIM, DAMAGES OR OTHER %%
%% LIABILITY, WHETHER IN AN ACTION OF CONTRACT, TORT OR OTHERWISE, ARISING    %%
%% FROM, OUT OF OR IN CONNECTION WITH THE SOFTWARE OR THE USE OR OTHER        %%
%% DEALINGS IN THE SOFTWARE.                                                  %%
%%                                                                            %%
%%                                                                            %%
%%%%%%%%%%%%%%%%%%%%%%%%%%%%%%%%%%%%%%%%%%%%%%%%%%%%%%%%%%%%%%%%%%%%%%%%%%%%%%%%
%%%%%%%%%%%%%%%%%%%%%%%%%%%%%%%%%%%%%%%%%%%%%%%%%%%%%%%%%%%%%%%%%%%%%%%%%%%%%%%%

%\documentclass[english, 12pt, a4paper, elec, utf8, pdfa]{aaltothesis}
\documentclass[english, 12pt, a4paper, elec, utf8, pdfa, online]{aaltothesis}
%\documentclass[english,12pt,a4paper,dvips]{aaltothesis}

%% Use the following options in the \documentclass macro above:
%% your school: arts, biz, chem, elec, eng, sci
%% the character encoding scheme used by your editor: utf8, latin1
%% thesis language: english, finnish, swedish
%% make an archiveable PDF/A compatible file: pdfa
%% symmetric typeset layout and blue hypertext for online publication: online
%%            (no option is the default, resulting in a wide margin on the
%%             binding side of the page and black hypertext)
%% two-sided printing: twoside (default is one-sided printing)
%%

\usepackage{graphicx}
\usepackage{cite}
\usepackage{amsfonts,amssymb,amsbsy}

\degreeprogram{Automation and information technology}
\major{Automation and systems engineering}
\code{AUT}
\univdegree{BSc}

\thesisauthor{Roope Savolainen}
\thesistitle{Modeling and analyzing Helsinki's traffic network using a microscopic simulator}
\place{Espoo}
\date{<DATE>}
\supervisor{Prof.\ Pekka Forsman}
\advisor{Dr Themistoklis Charalambous}

\uselogo{aaltoRed}{''}

\keywords{traffic simulation\spc traffic flow theory}
\thesisabstract{
	Abstract here.
}

\copyrighttext{Copyright \noexpand\copyright\ \number\year\ \ThesisAuthor}
{Copyright \copyright{} \number\year{} \ThesisAuthor}

\begin{document}

\makecoverpage
\makecopyrightpage

\begin{abstractpage}[english]
	\abstracttext{}
\end{abstractpage}

\newpage

\thesistitle{Helsingin liikenneverkon mallintaminen ja analyysi mikroskooppisella simulaatiolla}
\advisor{FT Themistoklis Charalambous}
\degreeprogram{Automaatio- ja informaatioteknologia}
%\department{Elektroniikan ja nanotekniikan laitos}
\major{Automaatio- ja systeemitekniikka}
\keywords{liikennesimulaatio, liikennevirtateoria}

\begin{abstractpage}[finnish]
	Suomenkielinen tiivistelmä tähän.
\end{abstractpage}

\newpage

%% Preface
%%
%%\mysection{Preface}
%\mysection{Esipuhe}
%%I want to thank Professor Pirjo Professori and my instructor Dr Alan Advisor for
%%their good and poor guidance.\\

%%\vspace{5cm}
%%Otaniemi, <DATE>

%%\vspace{5mm}
%%{\hfill Roope A.\ Savolainen \hspace{1cm}}

%% Force a new page after the preface
%%
%%\newpage


\thesistableofcontents


%% Symbols and abbreviations
\mysection{Symbols and abbreviations}

TODO: I wonder if I should keep this section, since it's going to be very short.

\subsection*{Symbols}

\begin{tabular}{ll}
$Q$  & traffic flow rate  \\
$\rho$              & traffic density  \\
$v$              & velocity  \\
\end{tabular}

%%\subsection*{Operators}

%%\begin{tabular}{ll}
%%$\nabla \times \mathbf{A}$              & curl of vectorin $\mathbf{A}$\\
%%$\displaystyle\frac{\mbox{d}}{\mbox{d} t}$ & derivative with respect to
%%variable $t$\\[3mm]
%%$\displaystyle\frac{\partial}{\partial t}$  & partial derivative with respect
%%to variable $t$ \\[3mm]
%%$\sum_i $                       & sum over index $i$\\
%%$\mathbf{A} \cdot \mathbf{B}$    & dot product of vectors $\mathbf{A}$ and
%%$\mathbf{B}$
%%\end{tabular}

%%\subsection*{Abbreviations}

%%\begin{tabular}{ll}
%%AC         & alternating current \\
%%APLAC      & an object-oriented analog circuit simulator and design tool \\
%%           & (originally Analysis Program for Linear Active Circuits) \\
%%BCS        & Bardeen-Cooper-Schrieffer \\ %% dash between the names
%%DC         & direct current \\
%%TEM        & transverse eletromagnetic
%%\end{tabular}

\cleardoublepage

\section{Introduction}

\thispagestyle{empty}

Traffic jams are not only an annoyance in our everyday lives, but also a burden on the efficient functioning of the society. One of the basic tasks of a state is to build and maintain infrastructure that is sufficient to support the local economy, but the rapidly developing urban areas do not make it an easy one.

This thesis will examine Helsinki's traffic network. The city centre of Helsinki was originally built in the 19th century, but the population of the city has grown over ten-fold since then \cite{helsinki}. To cope with the resulting dramatic increase in traffic amounts traffic control methods are usually implemented instead of building new infrastructure. For example, congestion pricing models have been implemented in cities such as London and Stockholm to protect areas most susceptible to congestion \cite{congestionpricing}.

Traffic flow research attempts to model the behavior of real-life traffic networks. Theoretical, empirically verified models of traffic can be useful in comparing traffic control methods and determining optimal ways to control traffic and minimize congestion.

The goal of this thesis is to simulate Helsinki's traffic network using VisSim traffic simulation software and compare the effectiveness of different traffic control methods in minimizing traffic congestion. The thesis is divided into five sections. In the second section, the theoretical basis behind the used traffic simulation methods will be explored. The simulation methods and used material will be discussed in the third section. The results of the simulation will be presented in the fourth section. The fifth section contains a summary of the thesis.

\clearpage

\section{Background}

\subsection{Traffic flow}

The field of traffic flow examines vehicle traffic on road networks. There are several different models used in analyzing traffic flow. This thesis uses the microscopic and macroscopic models. The microscopic model is used to model the traffic in the developed simulation and values describing the macroscopic model will be derived from the simulation.

Microscopic traffic flow modelling simulates the behavior of each individual vehicle. The basis of most microscopic models is the car-following model. It models the behavior of vehicles in such a way that they try to keep a distance determined by the speed of the vehicle to the vehicle in front of them. On top of this, the models based on car-following can specify the acceleration behavior, braking strategy and other more sophisticated properties of the modelled vehicles.

As opposed to microscopic models, macroscopic traffic flow models do not consider individual vehicles, but model the vehicle flow similarly to liquids and gases. The important quantities in this model are the flow density $\rho$, flow rate $Q$ and vehicle velocity $v$. The quantities are related by the flow-density relation

\[ Q = \rho v .\]

These macroscopic quantities are useful in describing the congestion and throughput of the observed roads. Flow can be measured from a microscopic model by counting the number of vehicles passing a point on a road. Thus, if we know the average velocities of the vehicles, we can also calculate the density of the traffic in the model. \cite{treiber}

A useful tool of macroscopic traffic analysis is the fundemental diagram of traffic flow. The flow rate of traffic at a single point on a road can be considered as a function of the flow density. The maximum of this function is found at point $\rho_c$, which is called the critical density. After the density of traffic passes this point, congestion starts to form.\cite{kerner}

\subsection{Pathfinding}

TODO: Write about basic graph theory and pathfinding once progress has been done on the simulation and I know what needs to be explained. Maybe just graphs and Dijkstra?

\clearpage

\section{Research material and methods}

TODO: consider the following points

* The simulation will be implemented in the VISSIM software. Pedestrians will most likely be ignored to keep the focus. Also need to decide on speed limits. It will probably be a reasonable amount of work to categorize the roads into 2 or 3 speed limit categories.

* What kind of an area will be included in the simulation? What kinds of roads? Helsinki centre, but will we include highways such as Länäri as exit/entry points?

\clearpage

\section{Results}

TODO: Results here. Exact structure is hard to know at this point, but could look something like this.

\subsection{Simulation results}

TODO: Examine simulation outcomes one-by-one. Probably group by traffic control methods used, draw diagrams of traffic flow for each.

\subsection{Comparison of used models}

TODO: Compare simulations and their results. See which parameters lead to best results.

\clearpage

\section{Summary}

TODO: Summary here. Pretty self-evident.

\clearpage

\thesisbibliography

\bibliography{bibliography}{}
\bibliographystyle{ieeetr}

\clearpage

\thesisappendix

\end{document}
