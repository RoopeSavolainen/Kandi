%%%%%%%%%%%%%%%%%%%%%%%%%%%%%%%%%%%%%%%%%%%%%%%%%%%%%%%%%%%%%%%%%%%%%%%%%%%%%%%%
%%%%%%%%%%%%%%%%%%%%%%%%%%%%%%%%%%%%%%%%%%%%%%%%%%%%%%%%%%%%%%%%%%%%%%%%%%%%%%%%
%%                                                                            %%
%% thesistemplate.tex version 3.01 (2017/10/06)                               %%
%% The LaTeX template file to be used with the aaltothesis.sty (version 3.01) %%
%% style file.                                                                %%
%%                                                                            %%
%% This is licensed under the terms of the MIT license below.                 %%
%%                                                                            %%
%% Copyright 2017, by Luis R.J. Costa, luis.costa@aalto.fi,                   %%
%% Copyright 2017 documentation in Finnish in the template by Perttu Puska,   %%
%% perttu.puska@aalto.fi                                                      %%
%% Copyright Swedish translations 2014 by Elisabeth Nyberg,                   %%
%% elisabeth.nyberg@aalto.fi and Henrik Wallén, henrik.wallen@aalto.fi        %%
%%                                                                            %%
%% Permission is hereby granted, free of charge, to any person obtaining a    %%
%% copy of this software and associated documentation files (the "Software"), %%
%% to deal in the Software without restriction, including without limitation  %%
%% the rights to use, copy, modify, merge, publish, distribute, sublicense,   %%
%% and/or sell copies of the Software, and to permit persons to whom the      %%
%% Software is furnished to do so, subject to the following conditions:       %%
%% The above copyright notice and this permission notice shall be included in %%
%% all copies or substantial portions of the Software.                        %%
%% THE SOFTWARE IS PROVIDED "AS IS", WITHOUT WARRANTY OF ANY KIND, EXPRESS OR %%
%% IMPLIED, INCLUDING BUT NOT LIMITED TO THE WARRANTIES OF MERCHANTABILITY,   %%
%% FITNESS FOR A PARTICULAR PURPOSE AND NONINFRINGEMENT. IN NO EVENT SHALL    %%
%% THE AUTHORS OR COPYRIGHT HOLDERS BE LIABLE FOR ANY CLAIM, DAMAGES OR OTHER %%
%% LIABILITY, WHETHER IN AN ACTION OF CONTRACT, TORT OR OTHERWISE, ARISING    %%
%% FROM, OUT OF OR IN CONNECTION WITH THE SOFTWARE OR THE USE OR OTHER        %%
%% DEALINGS IN THE SOFTWARE.                                                  %%
%%                                                                            %%
%%                                                                            %%
%%%%%%%%%%%%%%%%%%%%%%%%%%%%%%%%%%%%%%%%%%%%%%%%%%%%%%%%%%%%%%%%%%%%%%%%%%%%%%%%
%%%%%%%%%%%%%%%%%%%%%%%%%%%%%%%%%%%%%%%%%%%%%%%%%%%%%%%%%%%%%%%%%%%%%%%%%%%%%%%%

%\documentclass[english, 12pt, a4paper, elec, utf8, pdfa]{aaltothesis}
\documentclass[english, 12pt, a4paper, elec, utf8, pdfa, online]{aaltothesis}
%\documentclass[english,12pt,a4paper,dvips]{aaltothesis}

%% Use the following options in the \documentclass macro above:
%% your school: arts, biz, chem, elec, eng, sci
%% the character encoding scheme used by your editor: utf8, latin1
%% thesis language: english, finnish, swedish
%% make an archiveable PDF/A compatible file: pdfa
%% symmetric typeset layout and blue hypertext for online publication: online
%%            (no option is the default, resulting in a wide margin on the
%%             binding side of the page and black hypertext)
%% two-sided printing: twoside (default is one-sided printing)
%%

\usepackage{graphicx}
\usepackage{cite}
\usepackage{amsfonts,amssymb,amsbsy}

\degreeprogram{Automation and information technology}
\major{Automation and systems engineering}
\code{AUT}
\univdegree{BSc}

\thesisauthor{Roope Savolainen}
\thesistitle{Modeling and analyzing Helsinki's traffic network using a microscopic simulator}
\place{Espoo}
\date{<DATE>}
\supervisor{Prof.\ Pekka Forsman}
\advisor{Dr Themistoklis Charalambous}

\uselogo{aaltoRed}{''}

\keywords{traffic simulation\spc traffic flow theory}
\thesisabstract{
	Abstract here.
}

\copyrighttext{Copyright \noexpand\copyright\ \number\year\ \ThesisAuthor}
{Copyright \copyright{} \number\year{} \ThesisAuthor}

\begin{document}

\makecoverpage
\makecopyrightpage

\begin{abstractpage}[english]
	\abstracttext{}
\end{abstractpage}

\newpage

\thesistitle{Helsingin liikenneverkon mallintaminen ja analyysi mikroskooppisella simulaatiolla}
\advisor{FT Themistoklis Charalambous}
\degreeprogram{Automaatio- ja informaatioteknologia}
%\department{Elektroniikan ja nanotekniikan laitos}
\major{Automaatio- ja systeemitekniikka}
\keywords{liikennesimulaatio, liikennevirtateoria}

\begin{abstractpage}[finnish]
	Suomenkielinen tiivistelmä tähän.
\end{abstractpage}

\newpage

%% Preface
%%
%%\mysection{Preface}
%\mysection{Esipuhe}
%%I want to thank Professor Pirjo Professori and my instructor Dr Alan Advisor for 
%%their good and poor guidance.\\

%%\vspace{5cm}
%%Otaniemi, <DATE>

%%\vspace{5mm}
%%{\hfill Roope A.\ Savolainen \hspace{1cm}}

%% Force a new page after the preface
%%
%%\newpage


\thesistableofcontents


%% Symbols and abbreviations
%%\mysection{Symbols and abbreviations}

%%\subsection*{Symbols}

%%\begin{tabular}{ll}
%%$\mathbf{B}$  & magnetic flux density  \\
%%$c$              & speed of light in vacuum $\approx 3\times10^8$ [m/s]\\
%%$\omega_{\mathrm{D}}$    & Debye frequency \\
%%$\omega_{\mathrm{latt}}$ & average phonon frequency of lattice \\
%%$\uparrow$       & electron spin direction up\\
%%$\downarrow$     & electron spin direction down
%%\end{tabular}

%%\subsection*{Operators}

%%\begin{tabular}{ll}
%%$\nabla \times \mathbf{A}$              & curl of vectorin $\mathbf{A}$\\
%%$\displaystyle\frac{\mbox{d}}{\mbox{d} t}$ & derivative with respect to 
%%variable $t$\\[3mm]
%%$\displaystyle\frac{\partial}{\partial t}$  & partial derivative with respect 
%%to variable $t$ \\[3mm]
%%$\sum_i $                       & sum over index $i$\\
%%$\mathbf{A} \cdot \mathbf{B}$    & dot product of vectors $\mathbf{A}$ and 
%%$\mathbf{B}$
%%\end{tabular}

%%\subsection*{Abbreviations}

%%\begin{tabular}{ll}
%%AC         & alternating current \\
%%APLAC      & an object-oriented analog circuit simulator and design tool \\
%%           & (originally Analysis Program for Linear Active Circuits) \\
%%BCS        & Bardeen-Cooper-Schrieffer \\ %% dash between the names
%%DC         & direct current \\
%%TEM        & transverse eletromagnetic
%%\end{tabular}

\cleardoublepage

\section{Introduction}
	
\thispagestyle{empty}

Traffic jams not only are an annoyance in our everyday lives, but also a burden on the efficient functioning of the society. One of the basic tasks of a state is to build and maintain infrastructure that is sufficient to support the local economy, but the rapidly developing urban areas certainly don't make it an easy one.

In this thesis, we will be examining Helsinki's traffic network. While Helsinki's city centre was originally built in the 19th century, the population of the city has grown over ten-fold since then \cite{helsinki}. To cope with the resulting dramatic increase in traffic amounts, instead of building new infrastructure, traffic control methods are usually the most economical option. For example, congestion pricing models have been implemented in cities such as London and Stockholm to protect areas most susceptible to congestion \cite{congestionpricing}.

Traffic flow is a study that attempts to model the behavior of real-life traffic networks. Theoretical, empirically verified models of traffic can be useful in comparing traffic control methods and determining optimal ways to control traffic and minimize congestion.

\clearpage

\section{Background}

\subsection{Traffic simulation}

Traffic simulation theory here; the divide into microscopic and macroscopic models, their differences etc. Fundemental diagram of traffic flow.

\subsection{Pathfinding}

Basic theory behind pathfinding algorithms; graphs, Dijkstra, A-star.

\clearpage

\section{Research material and methods}

This section will be subsection-ed once I get the simulation stuff properly under started. Too hard to throw together a reasonable structure yet.

Some facts and open questions about the simulation:

The simulation will be implemented in the VISSIM software. Pedestrians will most likely be ignored to keep the focus. Also need to decide on speed limits. It will probably be a reasonable amount of work to categorize the roads into 2 or 3 speed limit categories even if we don't get that data automatically.

Where will map data come from? Self built vs automatically fetched from OpenStreetMaps etc? What kind of an area will be included in the simulation? What kinds of roads? Helsinki centre, but will we include highways such as Länäri as exit/entry points?

\clearpage

\section{Results}

Results here. Exact structure is hard to know at this point, but could look something like this.

\subsection{Simulation results}

Examine simulation outcomes one-by-one. Probably group by traffic control methods used, draw diagrams of traffic flow for each.

\subsection{Comparison of used models}

Compare simulations and their results. See which parameters lead to best results.

\clearpage

\section{Summary} 

Summary here. Pretty self-evident.

\clearpage

\thesisbibliography

\bibliography{bibliography}{}
\bibliographystyle{ieeetr}

\clearpage

\thesisappendix

\end{document}
